\Chapter{Gyakorlati alkalmazása}

Az éttermi rendelések hatékony kiszállítása elengedhetetlen manapság. Ezen hatékony kiszállítás egyik alappillérje a megfelelő út kiválasztása. A vevő leadja a rendelést, innentől fogva ő minél hamarabb szeretné megkapni az étteremből rendelt csomagját. Az étterem érdeke, hogy minél kevesebb utat megtéve minél gyorsabban ki tudja szállítani a rendeléseket a kívánt helyre/helyekre. Egy jól megtervezett út pozitív hatással bír mind a két tényezőre. Egy hatásos útvonal által a vevő gyorsabban jut hozzá a rendeléséhez, ezáltal például a vevőt nem várakoztatják meg és az étel is meleg lesz még. Az étteremre vonatkozóan a pozitív vásárlók még több vásárlást ígérnek. Ezáltal számára a jól megválasztott út rendkívűl fontos. Mindezek mellett az étteremre más pozitív pénzügyi hatással is lehet, gondolok én itt az üzemanyag kezdvező felhasználásától kezdve, a futárok jól megszervezett útjáig. Utobbi azért, fontos mert ezáltal több helyre ki tud szállítani egy adott futár, így több bevételt termel az adott étteremnek.

\begin{figure}[h!]
\centering
\includegraphics[scale=0.7]{images/delivery.jpg}
\caption{Éttermi rendelés szemléltetése pénzügyi haszonnal}
\label{fig:delivery}
\end{figure}