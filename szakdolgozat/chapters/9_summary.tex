\Chapter{Összefoglalás}

A dolgozatom témája az éttermi rendelések optimalizációja és szimulációja, amely segít meghatározni, az optimálus útvonalat a futároknak kiszállítás közben. Az ezekhez készült saját fejlesztésű szoftvereket elkészítése és működése került bemutatásra különböző paraméterezésekkel. A példákban szereplő szemléltető diagramok direkten az adott paraméterezett szituációhoz lettek generálva.

A szoftverek tökéletesre fejlesztése hosszú időt vehet igénybe. Társítani lehetne egy navigációs szoftverhez, aminél nem véletlenszerűen generált pontok lennének, hanem a feltérképezné az utakat és a szerint alakítaná ki a pontokat. További fejlesztési lehetőségként két verziót tudnék elképzelni; egy központi rendszert, amely az éttermekben lenne, és egy kliens alkalmazást, amely a futároknál. Az étteremben kiválasztanák, hogy milyen esetről van szó, milyen paraméterekkel, és az autómatikusan kiosztaná azt a kliensek között. A futároknál lévő mutatná vizuálisan a futár aktuális helyét és a helyes utat is, amin haladva a legkisebb távot teszi meg.

Ezen fejlesztéseket eszközölve az éttermek a folyamat során megfelelően tudják kihasználni a futárjaikat. Ezáltal a vevői elégedettség mellett időre és pénzügyi előnyökre tehetnek szert.

\newpage

\noindent \textbf{\Large Summary}

\bigskip

The topic of my thesis is the optimization and simulation of the delivery of restaurant orders, which helps to determine the optimal route for couriers during delivery. I presented the operation of self-developed software for situations with parameterized data. The illustrative diagrams in the examples were generated directly for the given situations. The development of a perfect software can take a long time in this scenario.

In a further development, the developed algorithms could be associated with a navigation software that would not use randomly generated points, instead it would map the routes and map the points accordingly. It would have two versions, a server one that would be in restaurants and a client that would be at couriers. In the restaurant, they would choose what case it is and with what parameters, and it would automatically allocate it among the client versions. A pointer at the couriers would visually show the current location of the courier on the map. In addition, the shortest route to the destination would be visible.

By making these improvements, the restaurants need to be able to take advantage of their couriers in the right way. This allows them to gain time and financial benefits in addition to customer satisfaction.