\Chapter{Összefoglalás}

A dolgozatomban témája az éttermi rendelések optimalizációja és szimulációja, amely segít meghatározni, az optimálus útvonalat a futároknak kiszállítás közben. Az esekhez készült saját fejlesztésű szoftvereket elkészítése és működése került bemutatásra paraméterezett adatokkal. A példákban szereplő szemléltető diagramok direkten az adott paraméterezett szituációhoz lettek generálva.\\
A szoftverek tökéletesre fejlesztése hosszú időd vehet igénybe. Társítani lehetne egy navigációs szoftverhez, ami által nem autómatikusan generált pontok lennének, hanem a feltérképezné az utakat és aszerint alakítaná ki a pontokat. Két verziója lenne, egy központi ami az éttermekben lenne, és egy kliens, ami a futároknál lenne. Az étteremben kiválasztanák, hogy milyen esetről van van szó és milyen paraméterekkel, és az autómatikusan kiosztaná azt a kliens verziók között. A futároknál lévő mutatná vizuálisan a futár aktuális helyét és a helyes utat is, amin haladva a legkisebb távot teszi meg. \\ 
Ezen fejlesztéseket eszközölve az éttermek a folyamat során megfelelően tudják kihasználni a futárjaikat. Ezáltal a vevői elégedettség mellett időre és pénzügyi javakra tehetnek szert.
\\
\\
\\

The topic of my thesis is the optimization and simulation of restaurant orders, which helps to determine the optimal route for couriers during delivery. I presented the operation of self-developed software for situations with parameterized data. The illustrative diagrams in the examples were generated directly for the given parameterized situation. Perfect software development can take a long time in this scenario. \\
It could be associated with a navigation software that would not automatically generate points, but would map the routes and map the points accordingly. It would have two versions, a server one that would be in restaurants and a client that would be at couriers. In the restaurant, they would choose what case it is and with what parameters, and it would automatically allocate it among the client versions. The pointer at the couriers would visually show the current location of the courier on the map. In addition, the shortest route to the destination would be visible. \\
By making these improvements, the restaurants need to be able to take advantage of their couriers in the right way. This allows them to gain time and financial benefits in addition to customer satisfaction.