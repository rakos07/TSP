\Chapter{Összefoglalás}

A dolgozatomban témája az éttermi rendelések optimalizációja és szimulációja, amely segít meghatározni, az optimálus útvonalat a futároknak kiszállítás közben. Az esekhez készült saját fejlesztésű szoftvereket elkészítése és működése került bemutatásra paraméterezett adatokkal. A példákban szereplő szemléltető diagramok direkten az adott paraméterezett szituációhoz lettek generálva.\\
A szoftverek tökéletesre fejlesztése hosszú időd vehet igénybe. Társítani lehetne egy navigációs szoftverhez, ami által nem autómatikusan generált pontok lennének, hanem a feltérképezné az utakat és aszerint alakítaná ki a pontokat. Két verziója lenne, egy központi ami az éttermekben lenne, és egy kliens, ami a futároknál lenne. Az étteremben kiválasztanák, hogy milyen esetről van van szó és milyen paraméterekkel, és az autómatikusan kiosztaná azt a kliens verziók között. A futároknál lévő mutatná vizuálisan a futár aktuális helyét és a helyes utat is, amin haladva a legkisebb távot teszi meg. \\ 
Ezen fejlesztéseket eszközölve az éttermek a folyamat során megfelelően tudják kihasználni a futárjaikat. Ezáltal a vevői elégedettség mellett időre és pénzügyi javakra tehetnek szert.