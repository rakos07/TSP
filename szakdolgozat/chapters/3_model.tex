\Chapter{A kiszállítási problémák általános modelljei}

A kiszállítási probléma vizsgálatához először annak általános modelljét kell megadni.
A vizsgált absztrakciós szintén a modell három alapvető eleme az étterem, a futár és a rendelés/kiszállítás helye (\ref{fig:generalModel}. ábra).

\begin{figure}[h!]
	\centering
	\includegraphics[scale=0.5]{images/complexModel.png}
	\caption{Éttermi rendelés modelljének három fő tényezője, úgy mint az étterem, a futár és a rendelés helye}
	\label{fig:generalModel}
\end{figure}

Az absztrak modellek nyilván nem írják le teljes részletességében a problémákat.\cite{Image1} \cite{Image2} \cite{Image3}
Jelen esetben a következőket várjuk el a felírt modelltől.
\begin{itemize}
	\item Az éttermeket és a kiszállítások helyét pontszerűnek tekinti.
	\item Szóhasználatot illetően azt feltételezzük, hogy a kiszállítás egyetlen városon belül történik. (Ez a matematikai modell szempontjából nem lényegi megkötés, de a problémák leírását egyszerűsíti.)
	\item Nem feltételezi, hogy a futárnak lenne kapacitás, üzemanyag, vagy bármilyen hasonló jellegű limitációja.
	\item A kiszállítások bejárási sorrendjére vonatkozóan azon túlmenően, hogy mindegyiket be kell járnia a futárnak, külön nincsen. Az optimalizálási probléma eredményeként várjuk, hogy mi az optimális sorrend.
	\item A kiszállítás pontos idejére vonatkozóan nincs korlátozó tényező. A felvázolt modell az időt csak a megtett út függvényében képes tekinteni.
	\item Nem tekintünk a modell részének semmilyen közbeiktatott változást (például a rendelés lemondását) vagy egyéb problémát (például a futár járművének meghibásodását, kiszállítással kapcsolatos problémát).
\end{itemize}
Ezen egyszerűsítések mellett is láthatjuk majd, hogy igen változatos esetekben, komoly optimalizálási problémák megoldására lesz szükség.

\Section{A probléma modelljének lehetséges esetei}

A szóbajöhető lehetőségek egyszerűbb áttekinthetősége érdekében adjunk meg hármasokat, melyekben az elemek az éttermek, futárok és kiszállítások számosságára vonat\-koznak.
A számosság itt lehet jelenthet egyet vagy többet. Előbbit $1$-el, utóbbit pedig $*$-al fogjuk jelölni.
Ezen jelölésrendszert használva összesen 8 lehetséges hármas adódna, úgy mint
\[
(1, 1, 1),
(1, 1, *),
(1, *, *),
(1, *, 1),
(*, 1, *),
(*, *, 1),
(*, 1, 1),
(*, *, *).
\]
A következő szakaszban azt vizsgáljuk és indokoljuk meg, hogy melyek azok az esetek, amelyekkel ezek közül nem érdemes foglalkozni.

\Section{Figyelmen kívül hagyott és vizsgált esetek}

\begin{itemize}
\item \textit{Egy étterem, több futár, egy kiszállítás}:
Nincs értelme vizsgálni, mivel egyetlen kiszállításhoz elegendő csak egy futár.
\item \textit{Több étterem, egy futár, egy kiszállítás}: Egy kiszállításnak szükségszerűen egy adott étteremből kell indulnia, tehát ez az eset nem értelmezhető.
\item \textit{Több étterem, több futár, egy kiszállítás}: Hasonlóan az előző esethez, nem tudjuk értelmezni, mert egy kiszállításhoz egyértelműen tartozik egyetlen étterem és egyetlen futár.
\end{itemize}

Az említett eseteket kihagyva az optimalizálási problémákat a következőkre tudjuk megadni:
\begin{itemize}
\item egy étterem, egy futár, egy kiszállítás,
\item egy étterem, egy futár, több kiszállítás,
\item több étterem egy futár, több kiszállítás,
\item egy étterem, több futár, több kiszállítás,
\item több étterem, több futár, több kiszállítás.
\end{itemize}
Az így adódó optimalizálási problémákat és azok lehetséges megoldásait a dolgozat a további fejezetekben mutatja be.

\Section{A város reprezentálása}

Úgy tekintjük, hogy az éttermek és a kiszállítások helyei is egy városon belül vannak.
A modell síkban gondolkozik, tehát egy étteremhez és egy kiszállítási helyhez is egy $(x, y) \in \mathbb{R}^2$ koordináta tartozik.

A közlekedési hálózat modellje egy gráf, amely azt adja meg, hogy melyik pontból melyikbe lehet eljutni.
Egy kiszállítás útvonala tehát ezen pontok közötti szakaszok sorozatának tekinthető, melyen az éttermek és a kiszállítási helyek is egy-egy pontot jelentenek.
