\Chapter{A kiszállítási problémák általános modelljei}

\Section{Összes eset}

Összes eset:

étterem: egy - több
futár: egy - több
kiszállítás: egy - több

(1, 1, 1)
(1, 1, *)
(1, *, *)
(1, *, 1)
(*, 1, *)
(*, *, 1)
(*, 1, 1)
(*, *, *)

\includegraphics[scale=0.5]{images/complexModel.png} \\

\Section{Kizárt esetek}

\subsection{
Egy étterem, több futár, egy kiszállítás
}

Nem opcionális eset, mivel a futárok közül csak az egyik dolgozna, hiszen csak egy kiszállítási hely van. Addig a többi futár csak várna.

\subsection{
Több étterem, egy futár, egy kiszállítás
}

Kizárt eset, hiszen több étterem megléte felesleges opcionalizálási célból, ha csak egy kiszállítási hely lett megadva.

\subsection{
Több étterem, több futár, egy kiszállítás
}

Az utóbbi két esetből mind a kettő olyan eset, ami nem életszerű. A jelenlegi eset szintén nem az, mivel a már említett esetek rossz tulajdonságait foglalja magába. \\ Több futár - Egy kiszállítási hely: nem valós, hiszen a futárok közül akkor csak egy dolgozna. \\ Több étterem - Egy kiszállítási hely: Felesleges több étterem egy kiszállítási helyhez.

\Section{Vizsgált esetek}

Egy étterem, egy futár, egy kiszállítás\\
Egy étterem, egy futár, több kiszállítás\\
Több étterem egy futár, több kiszállítás\\
Egy étterem, több futár, több kiszállítás\\
Több étterem, több futár, több kiszállítás\\
Ezen eseteket optimalizálási célból dolgozatomban a későbbiekben fejtem ki.
