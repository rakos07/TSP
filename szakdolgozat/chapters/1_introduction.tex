\Chapter{Bevezetés}

Az éttermi rendelések kiszállításánál megjelenő optimalizálási problémák egészen visszavezethetők az utaző ügynök problémájához. Az utazó ügynök problémája (Travelling salesman problem, TSP) egy kombinatorikus optimalizálási probléma. Az utazó ügynök problémának több variánsa van. Például az egy ügynökös, egy lerakatostól egészen a több ügynökös több lerakatos utazó ügynök problémáig. Maga a probléma NP nehéz, emiatt több fajta heurisztikus módszer született ennek megoldására. Klasszikus utazó ügynök probléma: adott n darab pont és páronként az egymástól való távolságuk. Egy ügynök mindegyik pontot meg akarja látogatni (tetszőleges pontból kiindulva), de mindegyiket csak egyszer. A feladat az, hogy határozzunk meg egy olyan körutat, amely minimális hosszúságú legyen. Ha feltesszük, hogy bizonyos pontokból nem lehet közvetlenül eljutni egyes pontokba, míg a többi pontba egységnyi költséggel lehet eljutni és az ügynöknek minden pontot meg kell látogatnia.
