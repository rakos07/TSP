\Chapter{Bevezetés}

Az éttermi rendelések kiszállításánál megjelenő optimalizálási problémák egészen visszavezethetők az utaző ügynök problémájához. Az utazó ügynök problémája (Travelling salesman problem, TSP) egy kombinatorikus optimalizálási probléma. Kiváló példa a bonyolultság-elmélet által NP-nehéznek nevezett problémaosztályra. Az utazó ügynök problémájához kapcsolódó matematikai feladatokkal először Sir William Rowan Hamilton és Thomas Penyngton Kirkman foglalkoztak az 1800-as években. Adott n darab pont és páronként az egymástól való távolságuk. Egy ügynök mindegyik pontot meg akarja látogatni (tetszőleges pontból kiindulva), de mindegyiket csak egyszer. A feladat az, hogy határozzunk meg egy olyan körutat, amely minimális hosszúságú legyen. Ha feltesszük, hogy bizonyos pontokból nem lehet közvetlenül eljutni egyes pontokba, míg a többi pontba egységnyi költséggel lehet eljutni és az ügynöknek minden pontot meg kell látogatnia.
