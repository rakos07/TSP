\Chapter{Bevezetés}

Az éttermi rendelések kiszállításánál megjelenő optimalizálási problémák igen széleskörűek. A mai modern technológia már lehetőséget ad ezen problémák megoldására. Az optimalizálási problémák megoldására számos különféle megoldás születt az évek alatt. \\
Szakdolgozatomban tárgyalt optimalizálandó problémákat a következőképpen lehet elkülöníteni: egy étterem, egy futár, egy cím; egy étterem, egy futár, több cím; több étterem, egy futár, több cím; egy étterem, több futár, több cím; több étterem, több futár, több cím. Ezen varációk mindegyikét külön problémaként kezelem, külön eljárással dolgozom ki a megoldásukat. \\
Az eseteket külön vizsgálva felírom a konkrét probléma megfogalmazását képpel illusztrálva. \\ 
Ezt követően rátérek a probléma megoldására. Dolgozatomban az optimalizálás célfüggvényét a megtett út hossza jelenti. A probléma bonyolultságát tekintve több algoritmust is felhasználtam az optimális útvonal meghatározásához. Ezen algoritmusok működését és a hozzájuk kapcsolódó fogalmakat alaposan részletezem a probléma megoldása közben. \\
Az esetek döntő többségéhez készült imlementáció is, ami a konkrét program kód fontosabb részeit tartalmazza magyarázattal. \\
Ezen programkódokat megadott paraméterekkel tesztelve, képekkel illusztrálva bizonyítom eredményességét.
