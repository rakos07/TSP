\Chapter{Bevezetés}

Az éttermi rendelések hatékony kiszállítása elengedhetetlen manapság. Ezen hatékony kiszállítás egyik alappillére a megfelelő út kiválasztása. A vevő leadja a rendelést, majd minél hamarabb szeretné megkapni az étteremből rendelt csomagját. Az étterem érdeke, hogy minél kevesebb utat megtéve minél gyorsabban ki tudja szállítani a rendeléseket a kívánt helyre/helyekre. Egy jól megtervezett út pozitív hatással bír mind a két tényezőre. Egy hatásos útvonal által a vevő gyorsabban jut hozzá a rendeléséhez, ezáltal például a vevőt nem várakoztatják meg, praktikusan az étel is meleg és friss lesz még. Az étteremre vonatkozóan a pozitív vásárlói vélemények még több vásárlást ígérnek. Így tehát a jól megválasztott útvonalak rendkívűl fontosak. Mindezek mellett az étteremre más pozitív pénzügyi hatással is lehet, gondolok itt az üzemanyag kezdvező felhasználásától kezdve, a futárok jól megszervezett útjáig. Utobbi azért fontos, mert ezáltal több helyre ki tud szállítani egy adott futár, így több bevételt termel az adott étteremnek.

Az éttermi rendelések kiszállításánál megjelenő optimalizálási problémák igen szé\-les\-körűek. A mai modern technológia már lehetőséget ad ezen problémák megoldására, ezekre számos különféle megoldás születt az évek alatt.

A szakdolgozatomban tárgyalt optimalizálandó problémákat a szerint lehet elkülöní\-teni, hogy mennyi az éttermek, futárok és a kiszállítások száma. Ezen varációk mindegyikét külön problémaként kezelem, külön eljárással dolgozom ki a megoldásukat.

Az eseteket külön vizsgálva mindegyik vizsgált esetben felírom a konkrét probléma megfogalmazását képpel illusztrálva.

Ezt követően rátérek a probléma megoldására. Dolgozatomban az optimalizálás célfüggvényét a megtett út hossza jelenti. A probléma bonyolultságát tekintve több algoritmust is felhasználtam az optimális útvonal meghatározásához. Ezen algoritmusok működését és a hozzájuk kapcsolódó fogalmakat alaposan részletezem a probléma megoldása közben.

Az esetek döntő többségéhez készült imlementáció is, ami a konkrét program kód fontosabb részeit tartalmazza magyarázattal.
Ezen programkódokat megadott paramé\-te\-rekkel tesztelve, képekkel illusztrálva látom be eredményességüket.
