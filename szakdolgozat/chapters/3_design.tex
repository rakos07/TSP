\Chapter{Tervezés}

Itt kezdődik a dolgozat lényegi része, úgy értve, hogy a saját munka bemutatása.
Jellemzően ebben szerepelni szoktak blokkdiagramok, a program struktúrájával foglalkozó leírások.
Ehhez célszerű UML ábrákat (például osztály- és szekvenciadiagramokat) használni.

Amennyiben a dolgozat inkább kutatás jellegű, úgy itt lehet konkretizálni a kutatási módszertant, a kutatás tervezett lépéseit, az indoklást, hogy mit, miért és miért pont úgy érdemes csinálni, ahogyan az a későbbiekben majd részletezésre kerül.

Ebben a fejezetben az implementáció nem kell, hogy túl nagy szerepet kapjon.
Ez még csak a tervezési fázis.
(Nyilván ha olyan a téma, hogy magának az implementációnak a módjával foglalkozik, adott formális nyelvet mutat be, úgy a kódpéldákat már innen sem lehet kihagyni.)

\Section{Táblázatok}

Táblázatokhoz a \texttt{table} környezetet ajánlott használni.
Erre egy minta \aref{tab:minta}. táblázat.
A hivatkozáshoz az egyedi \texttt{label} értéke konvenció szerint \texttt{tab:} prefixszel kezdődik.

\begin{table}[h]
\centering
\caption{Minta táblázat. A táblázat felirata a táblázat felett kell legyen!}
\label{tab:minta}
\begin{tabular}{l|c|c|}
a & b & c \\
\hline
1 & 2 & 3 \\
4 & 5 & 6 \\
\hline
\end{tabular}
\end{table}

\Section{Ábrák}

Ábrákat a \texttt{figure} környezettel lehet használni.
A használatára egy példa \aref{fig:cimer}. ábrán látható.
Az \texttt{includegraphics} parancsba 
Az ábrák felirata az ábra alatt kell legyen.
Az ábrák hivatkozásához használt nevet konvenció szerint \texttt{fig:}-el célszerű kezdeni.

\Section{További környezetek}

\subsection{
Pitagorasz tétel
}

A derékszögű háromszög befogóira emelt négyzetek területeinek összege egyenlő az átgofóra emelt négyzet területével.

\subsection{
Euklideszi távolság
}

Pitagorasz tételén alapul. Két pont távolsága a pontok különbségének, a pontok különbségének négyzetének gyökével egyenlő.

\subsection{
A* algorizmus
}

f(n) = g(n) + h (n) \\
g(n) - a kezdőpontból n-ig tartó út költsége \\
h(n) - n-től a célig vezető legolcsóbb út költése 

\subsection{
Gibbs faktor
}

\subsection{
A klasszikus utazó ügynök probléma matematikai megfogalmazása kiszállítási kritériumokra levetítve
}

Az egy ügynökös utazó ügynök probléma esetén jelölje V a csúcsok (pontok) halmazát, xi,j azt, hogy az i. pontból megy-e közvetlenül út a j. pontba. Az xi,j 1, ha útvonal köti össze a két pontot, különben 0:

A dij jelöli az i. és a j. pont távolságát, n pedig a pontok számát. A célfüggvény az alábbi:

A célfüggvénnyel magát a megtett távolságot szeretnénk optimalizálni. A pontba csak egy él fut be, tehát

Valamint, minden pontba csakis egy él távozik, ezek szerint

A sorrendiség a következő feltétel alapján érényesül

Itt ui az i. pont, uj a j. pont látogatási indexe, ahol az i. pontot hamarabb keresi fel az futár mint a j.-ot

\subsection{
A több ügynökös, egy lerakatos utazó ügynök probléma modellje kiszállítási helyzetekre szabva
}

A több ügynökös, egy lerakatos utazó ügynök probléma esetén legyen V a csúcsok halmaza, xi,j az, hogy megy-e az i. pontból út a j. pontba közvetlenül, di,j az i. és j. pont távolsága. Legyen m az ügynökök száma és ezáltal a kapott célfüggvény:

Minden pontba csakis egy út indul, kivétel ez alól a 0. pont ami maga az étterem:

Minden pontból csak egy út érkezik, kivétel ez alól a 0. pont ami maga az étterem:

Az étteremre való feltétel:

MTSP esetén a kövezkező megszorítások szükségessége elengedhetetlen a helyes sorrendmátrix beteljesüléséhez:

A definícióban S a pontok egy részhalmaza, r(S) pedig az, hogy ezt a részthalmazt minimum hány futárnak kell látogatnia. A definíció szumma része megmutatja, hogy minimum hány út megy be a vizsgált pontba. Az éttermet figyelmen kívűl hagyjuk m darab út hagyja el és m darab út megy ki belőle. Ezáltal a fenti egyenlőtlenség nem a futárok tényleges számát adná vissza. 
